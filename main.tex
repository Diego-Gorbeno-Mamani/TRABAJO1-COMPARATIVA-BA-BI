\documentclass{article}
\usepackage[utf8]{inputenc}
\usepackage{graphicx}
\graphicspath{ {./images/} }
\usepackage{multicol}
\usepackage[spanish, english]{babel}
\usepackage[left=3cm,right=3cm,top=3cm,bottom=3cm]{geometry}

\providecommand{\keywords}[1]{
  \small	
  \textbf{\textit{\quad \quad Keywords: }} #1}

\providecommand{\pclave}[1]{
  \small	
  \textbf{\textit{\quad \quad Palabras Clave: }} #1}

%Idiomas: \selectlanguage{english} \selectlanguage{spanish}

\begin{document}

\title{Trabajo Encargado N°1: COMPARATIVE BUSINESS ANALYTICS VS BUSINESS INTELLIGENCE}
\begin{titlepage}
\begin{figure}[htb]
\begin{center}
\includegraphics[width=5cm]{logo.png}
\end{center}
\end{figure}
\vspace*{-0.25in}
\begin{center}
\large{UNIVERSIDAD PRIVADA DE TACNA}\\
\vspace*{-0.025in}
INGENIERIA DE SISTEMAS  \\

\vspace*{0.5in}
\begin{large}
TITULO:\\
\end{large}

\vspace*{0.1in}
\begin{Large}
\textbf{COMPARATIVE BUSINESS ANALYTICS VS BUSINESS INTELLIGENCE} \\
\end{Large}

\vspace*{0.3in}
\begin{Large}
\textbf{CURSO:} \\
\end{Large}

\vspace*{0.1in}
\begin{large}
INTELIGENCIA DE NEGOCIOS\\
\end{large}

\vspace*{0.3in}
\begin{Large}
\textbf{DOCENTE:} \\
\end{Large}

\vspace*{0.1in}
\begin{large}
 Ing. Patrick Cuadros Quiroga\\
\end{large}

\vspace*{0.2in}
\vspace*{0.1in}
\begin{large}

Integrantes: \\
\begin{flushleft}
Briset Celia Garcia Salazar\hfill(2018062496) \\
Diego Manuel Gorbeño Mamani\hfill(2018000354)\\
Luis Fernando Flores Querie\hfill(2018062394)\\

\end{flushleft}
\end{large}

\vspace*{0.1in}
\begin{large}
Tacna - Perú\\
2022
\end{large}
\end{center}
\end{titlepage}

\vspace*{\fill}
	
\begin{center}
    
\selectlanguage{english}
\begin{abstract}
This report serves as a knowledge base for BI and BA solutions. To make smarter business decisions, identify issues, and maintain profitability, it's crucial to use digital tools and solutions to turn your data into actionable insights.

This report seeks to generate clear ideas by defining each concept and comparing these data management solutions, in order for organizations to obtain valuable information on industry trends and allow them to better focus on making strategic decisions.
\quad 

 \end{abstract}
 
 
 \keywords{}
 
 BI = Business Intelligence
 
BA = Business Analytics
\end{center}
\vspace*{\fill}

\newpage 


\begin{multicols}{2}

\section{Introduction}

Both business intelligence and business analytics give companies the ality to analyze data in order to make more informed decisions. The best strategy is often starting with a business intelligence program and then incorporating business analytics to make projections aimed at improving efficiency, revenue generation, etc. moving forward.
Understanding the differences between business intelligence and business analytics can help leaders choose the appropriate tools and talent to help grow their businesses. 


\section{Business Analytics}
It's used to examine the data and performance of an organization. It's a way to gain insight and make data-driven decisions in the future using statistical analysis.
The objective is to determine which data sets are useful and which can increase revenue. As well as productivity and efficiency.


\subsection{Types of Business Analytics}
\begin{itemize}
\item \textbf{ Descriptive Analytics}
It consists of preparing and analyzing historical data to identify patterns and trends.

\item \textbf{ Decision analytics}
Provide users with real-time insights on data related to current business operations. In this respect, decision analytics offer relevant insights for making timely decisions.

\item \textbf{ Predictive Analytics}
It allows determining the probality associated with future events based on the analysis of available information (present and past).

\item \textbf{ Prescriptive Analytics}
Determines new ways of operating that allow us to achieve our business objectives.
\end{itemize}


\subsection{Components of Business Analytics}
Each component is like an important gear in the machinery, and the successful running of analytics requires each component to work perfectly. Here are the various components of business analytics.
\begin{itemize}
\item \textbf{Data Aggregation}
Data is collected to one single, central location from where sorting can begin. Even duplicate data is checked for and removed completely. 
This data is taken from digital and physical forms that have been filled out by the consumers. The other form of data source is transaction data. 
\item \textbf{Data Mining}
To further go deep into the data, Data Mining is the next step to look for unknown patterns and trends.
Various statistics models are used.
\item \textbf{Association & Sequence Identification}
These components are a pattern of consumer behaviour.
This form of analytics components makes it easier to understand what the consumer is going to buy next and understand their buying patterns and behaviour.
\item \textbf{Text Mining}
What consumer types or comments in blogs and other social media comments or their interaction with customer service call centres are a part of the text mining component.
Helps in the development of new products based on the data collected, monitor competition and the developments they are making.
\item \textbf{Forecasting}
A famous saying goes that history repeats itself, and this saying is quite true when it comes to the forecasting components. It has been observed that consumers resort to a certain behaviour specific seasons or a period.
\item \textbf{Predictive Analytics}
Based on the data, one can create predictive analytics that will accurately make predictions in certain events.
This component of analytics helps companies understand what may happen in the near future and help them prepare for it.
\item \textbf{Optimization}
Business analytics really help with the optimization component. They can anticipate surges in demand and step production to maintain supply. They can competitively price their products when there is supposed to be a peak or shortage. Businesses can also create sales, offers, and discounts based on business analytics.
\item \textbf{Data Visualisation}
Data visualisation is one of the most effective ways of presenting data, and business analytics are quite helpful. This visual form of data helps companies make reports and sets new goals. 
\end{itemize}


\subsection{Business analytics challenges}
\begin{itemize}
\item Identifying the right data
\item Integrating the right data
\item Self-service challenges
\item Skills gap
\end{itemize}

\subsection{Business Analytics Tools}
\begin{itemize}
\item \textbf{Tableau}
 This business analytics tool is best known for presenting the fed data in attractive dashboards, charts and other forms of visuals mediums.
\item \textbf{Heap Analytics}
It helps you track and understand user behaviour that takes place in mole and web applications. 
\item \textbf{Alteryx}
 It supports many forms of data and you as the user can perform many different analytics. 
\item \textbf{MixPanel}
That helps you with retention of customers, predictive analytics and even can do A/B Testing, MixPanel is the tool for you. 
\item \textbf{Google Analytics}
It integrates well into other ecosystems. It's the worlds most used analytics tool owing to its ease, accessility, and costs.
\end{itemize}



\section{Business Intelligence}

Business intelligence () comnes business analytics, data mining, data visualization, data infrastructure and tools, and best practices to help organizations make more data-driven decisions.


\subsection{Business Intelligence Styles}
\begin{itemize}
\item \textbf{ Corporate or business reporting}
Reporters were used to generate static reports with a high degree of control over the visual format, intended for wide distribution to many people.

\item \textbf{ Cube Analysis}
Analytical capacity on a subset of data. Aimed at managers who require a safe and simple environment to explore and analyze a limited range of data.

\item \textbf{ Analysis and Ad Hoc queries}
Relational OLAP tools were used by advanced users to investigate and analyze the entire database, navigating to the most detailed level of information, that is, the transaction level.

\item \textbf{ Statistical analysis and data mining}
Application of mathematical, statistical and financial tools to find correlations, trends, projections and financial analysis. Aimed at advanced information analysts.

\item \textbf{ Delivery of information and alerts}
Information distribution engines were used to send reports or alarms to large groups of users, based on subscriptions, itineraries or events.
\end{itemize}

\subsection{Components of Business Intelligence}
The Business Intelligence infrastructure has three main components.

\begin{itemize}
\item Reporting Schema: The Cúram Business Intelligence and Analytics framework creates a domain-aware reporting schema that models the business processes of interest to Business Intelligence users. The reporting schema for the application consists of the Transition, Central, and Data Mart schemas. The application database is considered the source database or the operational database from which data is pulled.
\item Extraction processes: A set of extraction processes are provided in the Business Intelligence framework to populate the application's central data store and marts. These extract, transform, and load (ETL) processes are required to move the data from the application to the staging database. From there, they are moved to the central data warehouse (CDW) and finally to the data marts.
\item Built-in analytics: They are charts embedded in the online application that provide frontline business users with a view of the aggregated data in the data warehouse, helping them make decisions in their day-to-day business activities.

\end{itemize}

\subsection{Challenges when implementing Business Intelligence}

\begin{itemize}
\item Highest level sponsorship
\item resource constraint
\item Understanding of business rules
\item Quality of existing data
\item Justification of return on investment
\item departmental silos

\end{itemize}

\subsection{Business Intelligence Tools}
\begin{itemize}
\item \textbf{SAP Business Intelligence}
 SAP Business Intelligence offers several advanced analytics solutions including real-time  predictive analytics, machine learning, and planning and analysis.
\item \textbf{MicroStrategy}
MicroStrategy is a business intelligence tool that offers powerful (and high-speed) dashboarding and data analytics that help you monitor trends, spot new opportunities, improve productivity, and more. 
\item \textbf{Sisense}
 This user-friendly tool allows everyone within your organization to manage large and comprehensive data sets as well as analyze and visualize this data without the involvement of your IT department.
\item \textbf{Yellowfin }
Yellowfin  is an end-to-end business intelligence and analytics platform that comnes visualization, machine learning, and collaboration.
\end{itemize}
\section{Business Analytics vs Business Intellegence}
They function similarly but they have a few differences that can become deciding points for you to use either business analytics of business intelligence.\

When it comes to descriptive analysis and diagnostic analysis, business intelligence scores over business analytics as it provides a better look at the past and present condition of your business. However, when it comes to predictive analysis and prescriptive analysis, which show you the way forward, business analytics is better than business intelligence.


\section{Conclusions}
\begin{itemize}
    \item When it comes to business success, it's important to remember to use both solutions at the same time. Business Analytics and Business Intelligence solutions can take your business where you want it to go. All you need to know is how to take advantage of them for your organization.
    
    \item Both tools are complementary. They facilitate the systematic implementation of information cycles, from data collection to the interpretation of reports to act within the company, through data visualization systems.
    
\end{itemize}
 

\section{Recommendations}
\begin{itemize}
    \item The use of Business Intelligence is recommended if your company needs to understand internal operations, have control over what is happening, identify possible indicators or discover flaws in processes.
    
    \item The use of Business Analytics is recommended if your organization wants to anticipate possible changes in the market, predict future customer behavior or any other possible threat.
    
\end{itemize}
\end{multicols}
\newpage


\begin{thebibliography}

    \bibitem{DOC2008} Team ASM IBMR (2020, 2 de Marzo) Business Analytics (BA): Everything You Need to Know. https://www.asmibmr.edu.in/blog/business-analytics-ba-everything-you-need-to-know/
    
    \bibitem{FRE2016} 
   Galiana, P. (2021, 1 de Abril) Qué es Business Analytics: definición,tipos y diferencias. https://www.iebschool.com/blog/que-es-business-analytics-definiciontipos-y-diferencias-g-data/
   
    \bibitem{FRE2019} 
    Rojas, K. (2022, 10 de Febrero) ILEM - Diferencias entre g Data, Business Analytics y Business Intelligence. https://www.ilen.edu.pe/diferencias-entre-g-data-business-analytics-y-business-intelligence/
  
    \bibitem{FRE2019}
    McDermott, J. (2018, 23 de Agosto) Defining business analytics: an empirical approach.     https://www.tandfonline.com/doi/full/10.1080/2573234X.2018.1507605?scroll=top&needAccess=true
    
    \bibitem{FRE2018}
   Harvard Business Analytics Program (2012) Business Intelligence vs. Business Analytics. https://analytics.hbs.edu/blog/business-intelligence-vs-business-analytics/
   
   \bibitem{FRE2018}
   tableu (2018) ¿Qué es la inteligencia de negocios? Tu guía para la BI y por qué es importante. https://www.tableau.com/es-mx/learn/articles/business-intelligence
   
   \bibitem{FRE2018}
   SG (2012) Inteligencia de negocios. Los cinco estilos de BI. https://sg.com.mx/content/view/411
    
    \bibitem{FRE2018}
IBM Cúram Social Program Management (2021, 03 de Marzo) Componentes de Business Intelligence. https://www.ibm.com/docs/es/spm/7.0.4?topic=infrastructure-business-intelligence-components

    \bibitem{FRE2018}
INCAE (2017, 12 de Julio) Principales retos al implementar Business Intelligence. https://www.incae.edu/es/blog/2017/07/12/principales-retos-al-implementar-business-intelligence.html

    \bibitem{FRE2018}
Gilliam, Haije, E. (2019, 09 de Diciembre) Las 15 Mejores Herramientas de Inteligencia de Negocios: Una Vista General. https://mopinion.com/es/las-15-mejores-herramientas-de-inteligencia-de-negocios-una-vista-general/
    

 
    \end{thebibliography
    
    


\end{document}

